\documentclass{article}
\usepackage[a4paper, margin=2cm]{geometry}
\usepackage[utf8]{inputenc}
\usepackage[russian]{babel}
\usepackage{mathtools}
\usepackage{amsmath}

\title{ДЗ №1}
\author{Макаров Михаил}
\date{}

\begin{document}
	\maketitle
	\section{Задача №1}
	\begin{itemize}
	\item а) \( (a + 1)(ba)^* + (b + 1)(ab)^* \)
	\item б) Разобьём слово на максимальные блоки из букв b и c и блоки из букв a.
	Тогда два блока одного вида идти подряд не могут.\\
	h - блок из букв b и c, не включая пустое слово.
	\[ 
		h = (с(bc)^*(b + 1) + b(cb)^*(c + 1))
	\]
	Ответ:
	\[
		(a + 1)(ha)^* + (h + 1)(ah)^*
	\]
	\item в) Разобьём слово на блоки длины 2. Они бывают 4 видов (разбитых на 2 типа)
	\[
		c = \{ aa, bb \} = aa + bb
	\]
	\[
		d = \{ ab, ba \} = ab + ba
	\]
	Соответственно, правильные слова могут содержать сколько угодно блоков c и чётное число блоков d.
	Таким образом, язык сводится к языку над алфавитом $ \{c, d\} $, где буква d встречается в слове чётное число раз.
	Ответ:
	\[
		c^*((dc^*)^2)^* = (aa + bb)^*(((ab + ba)(aa + bb)^*)^2)^*
	\]
	\item г) Рассмотрим любое слово и допишем к нему с конца b. Тогда получившееся слово будет 
	словом из предыдущей задачи, но последний двубуквенный блок у него имеет вид ab или bb.
	Заметим, что в нём есть хотя бы 2 буквы b. Так как $ ab \in d $ и $ bb \in c $, то если полученное
	слово оканчивается на ab, то последний блок в нём типа d, иначе c.
	\[
		c^*((dc^*)^2)^*(dc^*ab + bb)
	\]
	Тогда исходный язык задаётся
	\[
		c^*((dc^*)^2)^*(dc^*a + b) = (aa + bb)^*(((ab + ba)(aa + bb)^*)^2)^*((ab + ba)(aa + bb)^*a + b)
	\]
	\item д) Выделим блоки из букв a в слове. До них могут идти либо ничего, если он в начале,
	либо буква b. Кроме того в слове есть буквы c и b.
	\[
		a^*(ba^* + c)^*
	\]
	\item е) Выделим блоки вида $ addd \ldots db$. По условию буквы a встречаются только в них, а на другие буквы ограничений нет.
	\[
		(b + c + d + ad^*b)^*
	\]
	\end{itemize}
	
	\section{Задача №2}
	Рассмотрим 2 случая: слово не содержит гласных (тогда это просто $c$),
	либо оно содержит гласные.\\
	Так как одновременно может быть только один вид гласных, то разберём только такой случай.\\
	Разобьём слово на максимальные блоки из гласных, которые заканчиваются согласной.
	Так как слово оканчивается на согласную и никакие 2 согласные не идут подряд,
	всё слово разобьётся на такие блоки кроме, может быть первой согласной. Тогда такие слова имеют вид
	\[
		(c + 1)(v^+c)^+
	\]
	Обьединяя все случаи вместе, получаем
	\[
		(c + 1)((v_1^+c)^+ + (v_2^+c)^+) + c
	\]
	
\end{document}
