\documentclass{article}
\usepackage[a4paper, margin=2cm]{geometry}
\usepackage[utf8]{inputenc}
\usepackage[russian]{babel}
\usepackage{mathtools}
\usepackage{amsmath}
\usepackage{tikz}

\usetikzlibrary{automata,positioning}

\title{ДЗ №4}
\author{Макаров Михаил}
\date{}

\begin{document}
	\maketitle

    \section{Задача №1}
    \subsection{А}
    \begin{tabular}{l}
    $ S \to AR $\\
    $ S \to LC $\\
    $ L \to aLb $\\
    $ R \to bRc $\\
    $ L \to aA $\\
    $ R \to cC $\\
    $ A \to aA $\\
    $ C \to cC $\\
    $ A \to \epsilon$\\
    $ C \to \epsilon$\\
    \end{tabular}\\
    По построению $L$ раскрывается в $a^nb^m|n > m$, $R$ в $b^nc^m|n < m$.
    \subsection{Б}
    \begin{tabular}{l}
    $ S \to aC $\\
    $ S \to AaaC$\\
    $ S \to CB$\\
    $ A \to aA$\\
    $ A \to \epsilon$\\
    $ B \to bB$\\
    $ B \to \epsilon$\\
    $ C \to aCb $\\
    $ C \to \epsilon$\\
    \end{tabular}\\
    По построению $C$ раскрывается в $a^lb^l$, тогда переходы из $S$ соответсвуют случаям, когда 
    $n - m = 1$, $n - m > 2$, $n - m < 0$.
    \subsection{В}
    \begin{tabular}{l}
    $ S \to CaCSCbC $\\
    $ S \to CbCSCaC$\\
    $ S \to CSCSC$\\
    $ S \to C$\\
    $ C \to cC $\\
    $ C \to \epsilon$\\
    \end{tabular}\\
    Слова из этого языка - суть слова из языка $\{ w \in \{a, b\}^* | |w|_a = |w|_b \} $, в которых ещё есть бувкы $c$.
    \subsection{Г}
    $ S \to \epsilon$\\
    $ S \to C$\\
    $ C \to CC $\\
    $ C \to ab $\\
    $ C \to aCCb $\\
    $C$ раскрывается в непустую подпоследовательность.
    \subsection{Д}
    $ S \to aSb | bSa $ \\
    $ S \to aC|Ca $ \\
    $ C \to aCb|bCa|\epsilon$\\
    Слова из этого языка - суть слова из языка $\{ w \in \{a, b\}^* | |w|_a = |w|_b \} $ с лишней буквой $a$. 
    Тогда переходы $ S \to aC|Ca $ соответсвуют добавлению лишней $a$ на каком-то этапе разбора слова.
    \section{Задача №2}
    Будем считать, что буквы $a$ увеличивают баданс на 2, $b$ снижают на 1.
    Построим грамматику так, что символ $U$ будет раскрываться в слова c неотрицательным балансом.\\
    $ U \to aUbUbU $\\
    $ U \to \epsilon $\\
    Действительно, $U$ может начинаться только на $a$. Так как $aUbUb$ соответствует словам с положительным балансом
    везде, кроме начала и конца, её позиция точно определена. Аналогично определим $N$ - слова с неположительным балансом.
    $ N \to \epsilon $\\
    $ N \to bNbNaN $\\
    Тогда раскрытия $S$ соответсвуют случаям, когда следующая позиция с нулевым балансом идёт после $a, b$.
    $ S \to \epsilon $\\
    $ S \to aUbUbS $\\
    $ S \to bNaUbS $\\
    $ S \to bNbNaS $\\
    Такая грамматика будет однозначной. Приведём её к нормальной форме:
    \begin{enumerate}
    \item Избавимся от переходов в символы алфавита:\\
    $ A \to a $\\
    $ B \to b $\\
    $ U \to AUBUBU $\\
    $ U \to \epsilon $\\
    $ N \to \epsilon $\\
    $ N \to BNBNAN $\\
    $ S \to \epsilon $\\
    $ S \to AUBUBS $\\
    $ S \to BNAUBS $\\
    $ S \to BNBNAS $\\
    \item Укоротим правила:\\
    $ A \to a $\\
    $ B \to b $\\
    $ U_1 \to BU$\\
    $ U_2 \to U_1U_1 $\\
    $ A_1 \to AU$\\
    $ N_1 \to BN$\\
    $ N_2 \to N_1N_1$\\
    $ A_0 \to AN$\\
    $ S_1 \to BS$\\
    $ S_0 \to AS$\\
    $ T \to A_1U_1$\\
    $ Y \to N_1A_1$\\
    $ U \to A_1U_2 $\\
    $ U \to \epsilon $\\
    $ N \to \epsilon $\\
    $ N \to N_2A_0 $\\
    $ S \to \epsilon $\\
    $ S \to TB_1 $\\
    $ S \to YB_1 $\\
    $ S \to N_2B_0 $\\
    \item Найдём $\epsilon$-порождающие: это $S, N, U$.\\
    $ S' \to S$
    $ S' \to \epsilon$
    $ A \to a $\\
    $ B \to b $\\
    $ U_1 \to BU$\\
    $ U_1 \to B$\\
    $ U_2 \to U_1U_1 $\\
    $ A_1 \to AU$\\
    $ A_1 \to A$\\
    $ N_1 \to BN$\\
    $ N_1 \to B$\\
    $ N_2 \to N_1N_1$\\
    $ A_0 \to AN$\\
    $ A_0 \to A$\\
    $ S_1 \to BS$\\
    $ S_1 \to B$\\
    $ S_0 \to AS$\\
    $ S_0 \to A$\\
    $ T \to A_1U_1$\\
    $ Y \to N_1A_1$\\
    $ U \to A_1U_2 $\\
    $ N \to N_2A_0 $\\
    $ S \to TB_1 $\\
    $ S \to YB_1 $\\
    $ S \to N_2B_0 $\\
    \item Уберём бессмысленные переходы:\\
    $ S' \to TB_1 $\\
    $ S' \to YB_1 $\\
    $ S' \to N_2B_0 $\\
    $ S' \to \epsilon$\\
    $ A \to a $\\
    $ B \to b $\\
    $ U_1 \to BU$\\
    $ U_1 \to b$\\
    $ U_2 \to U_1U_1 $\\
    $ A_1 \to AU$\\
    $ A_1 \to a$\\
    $ N_1 \to BN$\\
    $ N_1 \to b$\\
    $ N_2 \to N_1N_1$\\
    $ A_0 \to AN$\\
    $ A_0 \to a$\\
    $ S_1 \to BS$\\
    $ S_1 \to b$\\
    $ S_0 \to AS$\\
    $ S_0 \to a$\\
    $ T \to A_1U_1$\\
    $ Y \to N_1A_1$\\
    $ U \to A_1U_2 $\\
    $ N \to N_2A_0 $\\
    $ S \to TB_1 $\\
    $ S \to YB_1 $\\
    $ S \to N_2B_0 $\\
    \end{enumerate}
    
\end{document}
