\documentclass{article}
\usepackage[a4paper, margin=2cm]{geometry}
\usepackage[utf8]{inputenc}
\usepackage[russian]{babel}
\usepackage{mathtools}
\usepackage{amsmath}
\usepackage{amsfonts}
\usepackage{amsthm}

\title{ИПР №1 - 1.4.6b}
\author{Макаров Михаил}
\date{}

\newtheorem{lemma}{Лемма}

\begin{document}
	\maketitle

	\section{Доказательство}
	
	Множество векторов $ \{ v_1, \dots, v_m \}, v_i \in \mathbb{Z}_2^n $ называется линейно независимым, если 
	\[
		\forall a_i \in \mathbb{Z}_2, i = 1 \ldots m: \sum_{i=1}^{m} a_i v_i = 0 \leftrightarrow \forall a_i = 0
	\]
	Иначе множество $ V $ называется линейно зависимым. 
	Вектор x линейно выражается через $ V $, если
	\[
		\exists a_i \in \mathbb{Z}_2, i = 1 \ldots m: \sum_{i=1}^{m} a_i v_i = x
	\]
	Пусть $ L \subseteq \mathbb{Z}_2^n $ - произвольное линейное подпространство.\\
	\begin{lemma}
	В любом линейном подпространстве $L$ существует линейно независимое подмножество векторов
	$ V = \{ v_1, \ldots v_m \} $ такое, что любой вектор $ x \in L $ линейно выразим через $V$.
	\end{lemma}
	\begin{proof}
	Будем индуктивно строить линейно независимые множества $ V_i \subseteq L $ размера $i$. \\
	База: $ V_0 = \emptyset $ - очевидно линейно независимо. \\
	Переход: Пусть существует $ V_m = \{ v_1, \dots, v_m \} $. Если не существует вектора $x \in L$, который 
	линейно не выражается через $V_m$, то определим $ V = V_m $ и закончим индукцию. \\
	Иначе определим $ V_{m + 1} = V_m \cup \{ x \}, \; v_{m + 1} := x $.\\
	Предположим, что $ V_{m + 1} $ линейное зависимо, то есть 
	\[
		\exists a_i \in \mathbb{Z}_2, i = 1 \ldots m + 1 : \sum_{i=1}^{m + 1} a_i v_i = 0, \exists a_i \neq 0
	\]
	Если $ a_{m + 1} = 0 $, то верно
	\[
		\sum_{i=1}^{m} a_i v_i = 0
	\]
	откуда следует линейная зависимость множества $ V_i $, что противоречит предположнию индукции.\\
	Иначе $ a_{m + 1} = 1$, тогда
	\[
		\sum_{i=1}^{m} a_i v_i = x
	\]
	что означает, что x линейно выражается через $ V_i $. \\
	Следовательно, $ V_{m + 1} $ линейно независимо.\\
	Заметим, что так как $L$ конечно, то параметр $i$ не превосходит $|L|$, и, следовательно, множество $V$ определено.
	\end{proof}
	Возьмём множество $V, \; |V| = m $ из леммы для подпространства $L$. Предположим, что $\exists x \in L $ такой, что
	\[
		\exists a_i \in \mathbb{Z}_2, \exists b_i \in \mathbb{Z}_2, i = 1 \ldots m :
		\sum_{i=1}^{m} a_i v_i = \sum_{i=1}^{m} b_i v_i = x
	\]
	тогда
	\[
		\sum_{i=1}^{m} (a_i - b_i) v_i = 0
	\]
	из линейной независимости $V$ следует
	\[
		i = 1 \ldots m: a_i = b_i
	\]
	что означает, что любые две различные линейные комбинации векторов из $ V $ выражают разные векторы из $ L $.\\
	Рассмотрим фунцию $ f: \mathbb{Z}_2^m \rightarrow l $ такую, что
	\[
		f(y) = \sum_{i=1}^{m} y[i] * v_i
	\]
	где $ y[i] $ - $i$-я координата вектора $ y $.\\
	 По вышедоказанному 
	$ \forall x \in L \; \exists y \in \mathbb{Z}_2^m: f(y) = x$ и 
	$ \forall a, b \in \mathbb{Z}_2^m: a \neq b \rightarrow f(a) \neq f(b) $. Следовательно, $f$ - биекция, и 
	$ |L| = |\mathbb{Z}_2^m| = 2 ^ m $.
\end{document}
