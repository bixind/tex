\documentclass{article}
\usepackage[a4paper, margin=2cm]{geometry}
\usepackage[utf8]{inputenc}
\usepackage[russian]{babel}
\usepackage{mathtools}
\usepackage{amsmath}
\usepackage{tikz}
\usepackage{amsthm}

\usetikzlibrary{automata,positioning}

\newcommand{\range}[2]{
    \{#1, \ldots, #2 \}
}

\title{ИПР №5 - $3.1.1$}
\author{Макаров Михаил}
\date{}

\newtheorem{prop}{Утверждение}

\begin{document}
	\maketitle
    Пусть $G$ - произвольный граф, $V$ - множество его вершин, $E$ - множество его рёбер.
    \section{граф $G$ можно раскрасить в 2 цвета $\rightarrow$ граф $G$ двудольный}
    Пусть $G$ можно правильно раскрасить в 2 цвета $a, b$. Определим
    $A, B$ как множества вершин, раскрашенных в цвета $a, b$ соответственно. Тогда так
    как $A \cup B = V$, $A \cap B = \emptyset$, и 
    $ \forall (u, v) \in E: \neg (u \in A \land v \in A) \land \neg (u \in B \land v \in B) $, 
    то граф $G$ разбивается на доли $A, B$.
    \section{граф $G$ двудольный $\rightarrow$ граф $G$ содержит циклы только чётной длины}
    Пусть $A, B$ - доли графа $G$. Рассмотрим $c := v_1 e_1 \ldots e_l v_{l + 1}, v_{l + 1} = v_1$ - произвольный цикл     
    длины не меньше 2 в
    графе $G$. Предположим, что $l$ --- нечётно. Заметим, что 
    $ \forall i \in \range{1}{l}: \neg (v_i \in A \land v_{i + 1} \in A) \land \neg (v_i \in B \land v_{i + 1} \in B) $.
    Значит, для любого $i \in \range{1, l - 1}$ вершины $v_i, v_{i + 2}$ лежат в одной доле. Следовательно, так как
    $l$ - нечётно, вершины $v_1 = v_{l + 1}, v_l$ лежат в одной доле. Но тогда концы ребра $e_l$ лежат в одной доле.
    Противоречие, следовательно любой цикл в $G$ имеет чётную длину.
    \section{граф $G$ содержит циклы только чётной длины $\rightarrow$ граф $G$ можно раскрасить в 2 цвета}
    Пусть $G$ содержит циклы только чётной длины. Пусть $v$ - произвольная вершина графа $G$, 
    $V'$ - компонента связности, в которой лежит $v$. Зафиксируем для каждой вершины $u \in V'$ произвольный 
    путь $s_u := u_1 e^u_1 \ldots e^u_{l_u} u_{l_u + 1},u_1 = v, u_{l_u + 1} = u $ из $v$ в $u$. 
    Покрасим вершину $u$ в цвет $a$, если длина $s_u$ чётна, иначе в цвет $b$. \\
    Предположим, что 
    существуют две вершины $u \in V', w \in V'$ такие, что существует ребро $f = (u, w)$, и $u, w$ покрашены в один цвет.
    Тогда существует цикл $c:= v e^u_1 \ldots e^u_{l_u} u f w e^w_{l_w} \ldots e^w_1 v $ длины $l_u + l_w + 1$. 
    Так как $u, w$ покрашены в один цвет, то чётность чисел $l_w, l_u$ совпадает. Значит, длина $c$ нечётна, что 
    противоречит условиям. Следовательно, компоненту $V'$ можно покрасить в 2 цвета. Следовательно, любую компоненту
    $G$ можно покрасить в 2 цвета.\\
     Пусть $V'_1 \ldots V'_n$ --- компоненты $G$, $A'_1 \ldots A'_n$, $B'_1 \ldots B'_n$ ---
    множества вершин, покрашенных в цвет $a, b$ в компонентах $V'_1, \ldots, V'_n$ для их некоторых
    правильных раскрасок  соответственно. Определим множества $A := A'_1 \cup \ldots \cup A'_n$, 
    $B := B'_1 \cup \ldots \cup B'_n$. Тогда если покрасить вершины множеств $A, B$ в цвета $a, b$ соотвественно, то
    так как между компонентами связности нет рёбер, такая раскраска будет правильной.
\end{document}
