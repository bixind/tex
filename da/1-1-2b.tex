\documentclass{article}
\usepackage[a4paper, margin=2cm]{geometry}
\usepackage[utf8]{inputenc}
\usepackage[russian]{babel}
\usepackage{mathtools}
\usepackage{amsmath}

\title{ИПР №1 - 1.1.2b}
\author{Макаров Михаил}
\date{}

\begin{document}
	\maketitle

	\section{Доказательство}
	Разделим разбиения $ R_{n + 1} $ на $k + 1$ множество на 2 вида:
	\begin{itemize}
	\item Рассмотрим разбиения, содержащие множество $\{ n + 1 \}$.
		Таких разбиений \( \big\{ \substack{n \\ k} \big\} \), 
		поскольку если 	из любого из рассматриваемых разбиений $ R_{n + 1} $ на $k + 1$ множество вычесть $\{ n + 1 \}$,
		то полученное множество будет разбиением $ R_{n} $ на $k$ множеств, и, наоборот,
		если выбрать любое разбиение $ R_{n} $ на $k$ множеств, то при добавлении к нему множества $\{ n + 1 \}$
		полученное множество будет одним из рассматриваемых разбиений.
	\item Рассмотрим разбиения, не содержащие множество $\{ n + 1 \}$. Обозначим множество этих разбиений за $B$.
		Тогда $ n + 1 $ будет элементом хотя бы двухэлементного множества.
		Тогда при удалении этого элемента из этого множества получившееся разбиение
		будет разбиением $ R_{n} $ на $k + 1$ множество. Рассмотрим функцию $f: B \rightarrow C$, где 
		$C$ - множество разбиений $ R_{n} $ на $k + 1$ множество, которая удаляет элемент $n + 1$ из произвольного
		$b \in B$. Тогда так как у любого $c \in C$ существует ровно $k + 1$ прообраз из $B$, 
		то $|B| = (k + 1) * |C| = (k + 1) * \big\{ \substack{n \\ k + 1} \big\} $.
	\end{itemize}	 
	Таким образом \( \big\{ \substack{n + 1 \\ k + 1} \big\} = (k + 1) * \big\{ \substack{n \\ k + 1} \big\} + 
	\big\{ \substack{n \\ k} \big\} \).

\end{document}
