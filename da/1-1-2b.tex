\documentclass{article}
\usepackage[a4paper, margin=2cm]{geometry}
\usepackage[utf8]{inputenc}
\usepackage[russian]{babel}
\usepackage{mathtools}
\usepackage{amsmath}

\title{ИПР №1 - 1.1.2b}
\author{Макаров Михаил}
\date{}

\begin{document}
	\maketitle

	\section{Доказательство}
	Определим множество $ D_{n + 1, k + 1} = \{ W_i = \{ w_{i_1}, \ldots, w_{i_{l_i}} \} | i = 1, \ldots, k + 1,
	\bigsqcup_{i=1}^{k + 1} W_i = \mathcal{R}_{n + 1} \} $ - произвольное
	разбиение множества $\mathcal{R}_{n + 1}$ на k + 1 частей, $A_{n + 1, k + 1} = \{ D^j_{n + 1, k + 1} \} $ - 
	множество всевозможных разбиений $D_{n + 1, k + 1}$.\\
    Определим множество $B$: $B \subset A_{n + 1, k + 1}, D \in B \leftrightarrow \{ n + 1 \} \in D$.\\
    Определим функцию $f: B \rightarrow A_{n, k}, f(D) = D - \{ \{ n + 1 \} \}, D \in B$.
    Тогда $f^{-1}: A_{n, k} \rightarrow B, f^{-1}(D) = D \cup \{ \{n + 1\} \} $ - функция, обратная $f$.
    Действительно, $ f^{-1}(f(D)) = f^{-1}(D - \{ \{ n + 1 \} \}) = D$, так как по построению $B$,
    $ \forall D \in B: \{ n + 1 \} \in D$. С другой стороны, $f(f^{-1}(D)) = f(D \cup \{ \{n + 1\} \}) = D$.\\
    Таким образом, $f \circ f^{-1} = f^{-1} \circ f = id$, и, значит, $f$ - биекция. Следовательно
    $|B| = |A_{n, k}| = \left\{ \substack{n \\ k} \right\} $.\\
    Определим множество $C = A - B$. Тогда $ \forall D \in A_{n + 1, k + 1}: {n + 1} \notin D \leftrightarrow D \in C$ \\
    Определим функцию $g: C \rightarrow A_{n, k + 1},\; g(D) = \{W - \{n + 1\}| \forall W \in D\}, D \in C $.\\
    Тогда $\forall D' \in A_{n, k + 1} \forall i \in \{1, \ldots, k + 1\}:
     g((D' - \{ W_i \}) \cup \{ W_i \cup \{n + 1\} \}) = D'$. С другой стороны, \\
    $\forall D: f(D) = D' \; \exists i \in \{1, \ldots, k + 1\}: (D' - \{ W_i \}) \cup \{ W_i \cup \{n + 1\} \} = D$,
    так как $\exists! W \in D: n + 1 \in D$.\\
    Значит, у любого $D' \in A_{n, k + 1}$ существует ровно $k + 1$ прообраз из $C$. Следовательно, 
    $ |C| = (k + 1)|A_{n, k + 1}| = (k + 1) \left\{ \substack{n \\ k + 1} \right\}$. \\
    Так как $ A_{n + 1, k + 1} = B \sqcup C$, то 
    $|A_{n + 1, k + 1}| = |B| + |C| = \left\{ \substack{n + 1 \\ k + 1} \right\} = 
    (k + 1) \left\{ \substack{n \\ k + 1} \right\} + \left\{ \substack{n \\ k} \right\}$
    
	

\end{document}
