\documentclass{article}
\usepackage[a4paper, margin=2cm]{geometry}
\usepackage[utf8]{inputenc}
\usepackage[russian]{babel}
\usepackage{mathtools}
\usepackage{amsmath}
\usepackage{tikz}
\usepackage{amsthm}

\usetikzlibrary{automata,positioning}

\newcommand{\range}[2]{
    \{#1, \ldots, #2 \}
}

\title{ИПР №4 - $2.6.2c$}
\author{Макаров Михаил}
\date{}

\newtheorem{prop}{Утверждение}

\begin{document}
	\maketitle
	Пусть $G$ - произвольный граф, $V$ - множество его вершин, $E$ - множество его рёбер, $a_0 \ldots a_s$ -
	максимальный из путей в $G$, проходящих через каждую свою вершину один раз, $s \ge 3$, $ \deg a_0 + \deg a_s > s$.
    Обозначим $A := \{ a_i | i \in \range{0}{s} \}$.
    
    Предположим, что $ \exists v \in V: v \notin A \land ( (v, a_0) \in E \lor (v, a_s) \in E) $. Тогда 
    хотя бы один из путей $va_0 \ldots a_s$, $a_0 \ldots a_sv$ существует в $G$, проходит через все свои вершины
    один раз и длинее $a_0 \ldots a_s$, что противоречит максимальности $a_0 \ldots a_s$.

    Предположим, что в графе $G$ нет несамопересекающегося цикла длины $s + 1$.
    
    Пусть $(a_0, a_s) \in E$. Тогда существует несамопересекающегося цикл $a_0 \ldots a_s$ длины $s + 1$.
    
    Иначе вершины $a_0, a_s$ могут быть соеденены рёбрами только с вершинами $a_1 \ldots a_{s - 1} $. 
    Заметим, что если $ \exists i \in \range{1}{s - 2}: (a_0, a_{i + 1}) \in E \land (a_s, a_i) \in E$, то
    существует несамопересекающийся цикл $a_0\ldots a_i a_s \ldots a_{i + 1} $ длины $s + 1$. Предположим,
    что такого $i$ не существует. Определим $B := \{ a_i| i \in \range{2}{s} \land (a_{i - 1}, a_s) \in E \}$,
    обозначим через $C$ множество вершин, соединенных ребром с $a_0$.
    Тогда $|B| = \deg a_s$, $ C \cap B = \emptyset$. Тогда так как $C \subset A - \{ a_0 \}$,
    $B \subset A - \{ a_0 \}$, то $|C| + |B| \leq |A - \{ a_0 \}| = s$, откуда
    $deg a_0 + a_s \leq s$, что противоречит условиям. 
    
    Следовательно, в графе $G$ существует несамопересекающийся цикл длины $s + 1$.
    
	
\end{document}
