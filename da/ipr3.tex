\documentclass{article}
\usepackage[a4paper, margin=2cm]{geometry}
\usepackage[utf8]{inputenc}
\usepackage[russian]{babel}
\usepackage{mathtools}
\usepackage{amsmath}
\usepackage{tikz}
\usepackage{amsthm}

\usetikzlibrary{automata,positioning}

\newcommand{\range}[2]{
    \{#1, \ldots, #2 \}
}

\title{ИПР №3 - 2.2.1(3)}
\author{Макаров Михаил}
\date{}

\newtheorem{prop}{Утверждение}

\begin{document}
	\maketitle
	\section{Существование}
	Следует из того, что дерево - связный граф.
	\section{Единственность}
	Предположим, что существует два несамопересекающихся различных пути $s_1 := v_1e_1v_2\ldots v_n$,
	$s_2 := u_1f_1\ldots u_m$, такие что $u_1 = v_1, v_n = u_m$.\\
	Так как они различны, то
	$i_l := min\{i| i \in \range{1}{\min(n, m)}: v_i \neq u_i \lor e_i \neq f_i$\} определена корректно.\\
    \begin{prop} 
        $v_{i_l} = u_{i_l}$
    \end{prop}
    \begin{proof}
        При $i_l = 1$ утверждение верно, так как $u_1 = v_1$.
        Предположим что это не так при $i_l > 1$. Тогда $v_{i_l - 1}, u_{i_l - 1}$ существуют 
        и совпадают. Тогда так как 
        $v_{i_l - 1} = u_{i_l - 1}, v_{i_l} \neq u_{i_l}$, то $e_{i_l - 1} \neq f_{i_l - 1}$, что противоречит
        минимальности $i_l$.
    \end{proof}
    Следовательно, $e_{i_l} \neq f_{i_l}$.\\
    Определим $j_l = i_l$,\\
    $i_r := min\{i| i \in \range{i_l}{n}: \exists j \in \range{j_l}{m}: \neg(i = j = i_l) \land (v_i = u_j) \}$,\\
    $j_r := min\{j| j \in \range{j_l}{m}: \neg(i_r = j = i_l) \land (v_{i_r} = u_j) \}$.\\
    Заметим, что $i_r, j_r$ определены корректно, так как $v_n = u_m$.\\
    \begin{prop}
        1. $\forall i \in \range{i_l}{i_r} \forall j \in \range{j_l + 1}{j_r - 1}: u_{i} \neq v_{j}$\\
        2. $\forall i \in \range{i_l}{i_r - 1} \forall j \in \range{j_l}{j_r - 1}: e_{i} \neq f_{j}$
    \end{prop}
    \begin{proof}
        Предположим, что неверно 1. Тогда $\exists i_l \leq i \leq i_r, j_l < j < j_r: v_{i} = u_{j}$, что    
        противоречит минимальности выбора $i_l, j_l$.\\
        Предположим, что неверно 2. Тогда $\exists i_l \leq i < i_r, j_l \leq j < j_r: e_{i} = f_{j}$. Заметим, что
        так как $e_{i_l} \neq f_{i_l}$, то $\neg(i = j = i_l)$, что противоречит
        минимальности выбора $i_l, j_l$, так как $v_i = u_j$,.        
    \end{proof}
    Тогда из того, что $s1, s2$ не содержат самопересечений, утв. 2 и $\neg(i_r = j_r = i_l)$, следует, что цикл\\
     $c := (v_{i_l},e_{i_l}\ldots, v_{i_r} = u_{j_r}, f_{j_r - 1} \ldots u_{j_l} = v_{i_l})$ не 
    содержит самопересечений и состоит хотя бы из двух вершин,
    значит, $c$ - простой цикл, что противоречит тому, что граф - дерево.
    Следовательно, в дереве не существует двух различных путей между любыми двумя вершинами.
\end{document}
