\documentclass{article}
\usepackage[a4paper, margin=2cm]{geometry}
\usepackage[utf8]{inputenc}
\usepackage[russian]{babel}
\usepackage{mathtools}
\usepackage{amsmath}
\usepackage{tikz}
\usepackage{amsthm}

\usetikzlibrary{automata,positioning}

\newcommand{\range}[2]{
    \{#1, \ldots, #2 \}
}

\title{ИПР №3 - $2.2.1(2'')$}
\author{Макаров Михаил}
\date{}

\newtheorem{prop}{Утверждение}

\begin{document}
	\maketitle
    \section{Необходимость}
    Следует из определения дерева и задачи №$2.2.1(2)$.
    \section{Достаточность}
    Докажем это утверждение индукцией по числу вершин.\\
    \begin{itemize}
    \item База: $n = 1$.\\
    Существует единственный граф из 1 вершины. Так как в нём 0 рёбер, он связен и является деревом, то утверждение верно
    при $n = 1$.
    \item Шаг:\\
    Пусть $n > 1$. Рассмотрим произвольный связный граф $G$ с $n - 1$ ребром, обозначим $V$ - множество его вершин,
    $E$ - множество рёбер, $d(v), v \in V$ - степень вершины $v$. 
    Так как $G$ связен и $n > 1$, то степень каждой вершины больше 0.\\
    Предположим, что в $G$ не существует листа. Тогда степень каждой вершины хотя бы 2. Пусть $e$ - количество рёбер
    в $G$. Так как $e = \frac{1}{2} \sum\limits_{v \in V} d(v)$, то $e \ge \frac{1}{2} \sum\limits_{v \in V} 2 = n$, что 
    противоречит условию $e = n - 1$.\\
    Следовательно, в $G$ существует лист. Обозначим его $u$, вершину, соединённую с $u$ ребром - $v$.\\
    Пусть $G' := (V \setminus \{u\}, E \setminus \{ (u, v) \}) $. Тогда в $G'$ $n - 2$ ребра, $n - 1$ вершина.\\
    Предположим, что $G'$ несвязен: существуют вершины $a, b \in V \setminus \{u\}$ такие, что не существует пути 
    между $a$ и $b$. Тогда, в частности, $a \neq b$. Так как $G$ связен, $a, b \in V$, то существует 
    $c := v_1e_1\ldots e_{m - 1}v_m, \forall i \in \range{1}{m}: v_i \in V, \forall j \in \range{1}{m - 1}: e_j \in E, v_0 = a, v_m = b $ - минимальный по длине путь, соединяющий $a, b$.\\
     Заметим, что если $\forall i \in \range{1}{m}: v_i \in V \setminus \{u\}$, то 
     $\forall j \in \range{1}{m - 1}: e_j \in E \setminus \{ (u, v) \} $, и, значит,
     $c$ лежит в графе $G'$, что противоречит предположению об отсутствии пути между $a, b$. Следовательно, так как
     $a, b \in V \setminus \{u\}$, то 
     $\exists k \in \range{2}{m - 1}: v_k = u$. Так как $u$ - лист то $v_{k - 1} = v_{k + 1} = v$. Тогда путь
     $c' = v_1e_1\ldots v_{k-2}e_{k-2}ve_{k + 1} \ldots v_m$ соединяет вершины $a$ и $b$ в $G$ и короче чем $c$, что
     противоречит минимальности выбора $c$.\\
    Следовательно, $G'$ связен. По предположению индукции $G'$ - дерево.\\
    Предположим, что $G$ содержит несамопересекающийся цикл 
    $l := u_1f_1 \ldots f_{s - 1}u_s, \forall i \in \range{1}{s}: u_i \in V, \forall j \in \range{1}{s - 1}: f_j \in E,
    v_1 = v_s $. Тогда если $\forall i \in \range{1}{s}: u_i \in V \setminus \{u\}$, то $l$ полностью лежит в $G'$, что противоречит
    тому, что $G'$ - дерево. Следовательно $\exists i \in \range{1}{s}: u_i = u$. Тогда, считая что
    $f_{0} := f_{s - 1}$, и учитывая что $s > 2$, получаем $f_{i - 1} = f_{i}$. Значит, $l$ содержит самопересечения.\\
    Следовательно, $G$ не содержит несамопересекающихся циклов и является деревом.
    \end{itemize}
\end{document}
