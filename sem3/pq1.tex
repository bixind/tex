\documentclass{article}
\usepackage[a4paper, margin=2cm]{geometry}
\usepackage[utf8]{inputenc}
\usepackage[russian]{babel}
\usepackage{mathtools}
\usepackage{amsmath}
\usepackage{tikz}
\usepackage[]{algorithm2e}

\title{Q}
\author{Макаров Михаил}
\date{}

\begin{document}
	\maketitle
	\section{A}
    Рассмотрим потенциал $\Phi = max \{ h(v)| v \in V \setminus \{ s, t \}, e(v) > 0 \} + 1$, или, если
    такое множество пусто, $\Phi = 0$.
    Очевидно, в начале работы $\Phi = 1$ или $\Phi = 0$.\\
    На каждой итерации внешнего цикла верно одно из двух.
    \begin{itemize}
    \item Операция $RELABEL$ не была вызвана ни разу.
    Тогда высоты вершин не поменялись. Но тогда так как в вершины на высоте $\Phi$ не могло быть пущено потока,
    то после этой итерации для всех вершин на высоте $\Phi$ верно $e(v) = 0$, и, следовательно, 
    $\Phi$ уменьшился хотя бы на 1.
    \item Иначе высота хотя бы одной вершины увеличилась. Пусть $v$ - максимальная по высоте из таких вершин
    в конце итерации. Тогда $\Phi < h(v)$, так как вершины не ниже $v$ не могли стать снова переполнеными.
    Таким образом $\Phi \leq \Phi' + h(v) - h'(v)$, где штрихом помечены значения величин в начале итерации.
    \end{itemize}
    Таким образом, за всё время работы алгоритма потенциал суммарно мог увеличится на не более чем сумму финальных
    высот вершин. Так как $\Phi = 0$ в конце алгоритма итераций внешнего цикла 
    сделано не более чем количество вызовов операции $RELABEL$ и суммарного увеличения 
    потенциала суммарно, то есть $O(|V|^2)$, то есть количество вызовов операции 
    $DISCHARGE$ - $O(|V|^3)$, откуда асимптотика алгоритма - $O(|V|^3)$.
    \section{B}
    Перепишем алгоритм в виде:\\
    \begin{algorithm}[H]
        $queue := (V \setminus \{ s, t \})$\;
        \While{$queue$ not empty} {
            $newqueue := \{ \}$\;
            \For {$v \in queue$}{
                $DISCHARGE(v)$\;
                Добавить все вновь переполненные вершины в $newqueue$\;
            }
            $queue := newqueue$\;
        }
    \end{algorithm}
    Заметим, что в таком виде к алгоритму можно применить те же рассуждения, что и к
    алгоритму из пункта $A$. Тогда асиптотика - $O(|V|^3)$.
\end{document}
