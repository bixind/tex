\documentclass{article}
\usepackage[a4paper, margin=2cm]{geometry}
\usepackage[utf8]{inputenc}
\usepackage[russian]{babel}
\usepackage{mathtools}
\usepackage{amsmath}
\usepackage{tikz}

\title{B}
\author{Макаров Михаил}
\date{}

\begin{document}
	\maketitle
    \section{Существование идеального паросочетания $\rightarrow$ условие леммы Холла}
    Для любого $A$ рассмотрим множество вершин $B$, соединённых с вершинами множества 
    в идеальном паросочетании. Тогда $|B| = |A|, B \subset N(A) \rightarrow |A| \leq |N(A)|$.
    \section{Существование идеального паросочетания $\leftarrow$ условие леммы Холла}
    Докажем, что если нет идеального паросочетания, то условие леммы Холла не выполняется.
    Рассмотрим какое-либо максимальное паросочетание в $G$. Пусть $A^+, B^+$ - подмножества
    $L, R$ соотвественно, учавстующие в этом паросочетании, а $A^- = L \ A^+, B^- = R \ B^+$.\\
    Заметим, что так как паросочетание не идеальное, то $A^-$ непусто. 
    Рассмотрим множество вершин $T$, достижимых из множества $A^-$ по определённым правилам.
    А именно, из вершин левой доли разрешаются все переходы, а из вершин правой доли - только те,
    которые лежат в паросочетании. \\
    Докажем, что тогда $T \cap B^- = \emptyset$. Действительно, если существует путь
    $a_0, b_1, a_1, \ldots, a_k, b_{k + 1}, a_0 \in A^-, b_{k + 1} \in B^-$, то так как
    рёбра $b_i, a_i$ лежат в паросочетании, то избавляясь от них и добавляя в паросочетание
    рёбра $a_i, b_{i+1}$ мы построим паросочетание большего размера. Тогда $R \cap T \subset B^+$. \\
    Теперь заметим, что $ T \ A^- $ - это подмножество вершин,
    лежащих в паросочетании, причём вместе с каждой вершиной в нём лежит и вершина,
    соседняя по ребру в паросочетании. Обозначим $ A' = (T \ A^-) \cap A^+, B' = (T \ A^-) \cap B^+$.
    Из вышесказанного вытекает, что $|A'| = |B'|, T = A' \cup B' \cup A^-$. По построению
    $B' = N(A' \cup A^-)$, $ |A^-| > 0 \rightarrow |A^- \cup A'| > |A'| = |B'| = |N(A^- \cup A')|$.
    Значит, условие леммы Холла не выполняется.
\end{document}
