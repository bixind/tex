\documentclass{article}
\usepackage[a4paper, margin=2cm]{geometry}
\usepackage[utf8]{inputenc}
\usepackage[russian]{babel}
\usepackage{mathtools}
\usepackage{amsmath}
\usepackage{tikz}
\usepackage[]{algorithm2e}

\title{R}
\author{Макаров Михаил}
\date{}

\begin{document}
	\maketitle

    \section{A}
    Пусть ключевая вершина $r$ лежит в слое $V_l$, $h$ максимальный поток, который можно через неё пропустить.
    Предположим, что $h < \phi(r)$. Тогда, если пустить этот поток, не существует пути по ненасыщенным рёбрам
    из $s$ в $t$ через $r$. Рассмотрим множество $B$ достижимых из $r$ вершин, $r \in B$. Пусть 
    $v$ - такая вершина, что $v \in B, \; \forall u \in B: v \in V_a, u \in V_b \rightarrow a \geq b$.
    Заметим, что если $v = t$, то существует путь из $v$ в $t$.\\
    Иначе, так как $\phi(v) \geq \phi(r), v \neq t$, то существует ненасыщенное ребро из $v$, следовательно
    должна быть $w \in V_{a + 1}$, достижимая из $r$. Значит, такой вершины нет и $t$ достижима из $v$.
    Аналогично доказывается, что $r$ достижима из $s$, (так как пустить поток $h$ из $s$ в $r$ всё равно что
    пустить поток $-h$ из $r$ в $s$).\\
    Но тогда существует путь $s$ в $t$ через $r$, и, следовательно, поток $h$ не максимальный. Противоречие.
    Значит, через $r$ можно пустить поток $\phi(r)$.
    \section{B}
    Аналогично алгоритму Дейкстры будем итеративно искать блокирующий поток.\\
    На каждой итерации поиска блокируещего потока помимо слоистой сети будем считать
    списки входящих и исходящих рёбер в слоистой сети для каждой вершины.\\
    Будем итеративно находить ключевые вершины:
    \begin{enumerate}
    \item Сначала удалим все вершины с потенциалом 0. При этом учтём что все входящие и исходящие
    из них рёбра также исчезнут, пересчитаем потенциалы и повторим этот пункт, если появились
    новые нулевые вершины.\\
    Это можно сделать кладя вершины нулевого потенциала в очередь и помечая вершины как удалённые
    в отдельном массиве.\\
    Заметим, что в полученном графе $t$ достижима из любой вершины.
    \item Проталкивание: Найдём ключевую вершину $r$ (циклом по $V$) 
    и будем итеративно проводить поток $\phi(r)$ по слоям.
    Пусть $f'[v]$ - величина потока, втекающего в $v$:\\
        \begin{algorithm}[H]
            $f'[r] := \phi(r)$ \;
            \For{$i := l$ \KwTo $k$}{
                \For {$v \in N_v$} {
                    \For {$f'[v] > 0 \wedge u \in E_v$}{
                        $val := max(f'[v], c_f(v, u))$\;
                        пустить по $(u, v)$ поток $val$\;
                        $f'[v] -= val, f'[u] += val$\;
                    }
                }            
            }
        \end{algorithm}
        где $N_v$ - соседние вершины в следующем слое.\\
        Докажем, что после такой итерации $f'[v] > 0 \leftrightarrow v = t$.
        Действительно, пусть $v$ вершина, находящаяся в наибольшем слое среди всех таких вершин.
        Так как $\phi(v) \geq \phi(r)$ в частности выходная степень $v \geq \phi(r)$. Но тогда последний вложенный цикл
        должен был завершится из-за того, что $f'[v] = 0$ на очередной итерации. Противоречие.\\
        Выполним аналогичное проталкивание в сторону $s$, но только в этот раз поток будет втекать в $v$.
        Вернёмся к пункту 1.
    \end{enumerate}
     Оценим число операций. \\
     \begin{enumerate}
     \item Внешний цикл поиска блокирующего потока выполнит не более $|V|$ итераций, так как на каждой итерации
     потенциал хотя бы одной вершины обнуляется.
     \item По понятным причинам первый пункт сделает не больше $|V|$ итераций и просмотрит каждое ребро 2 раза.
     \item Проталкивание просмотрит не больше $|V|$ при каждом запуске.\\
        Заметим, что когда мы пускаем поток в операции проталкивания,
        то каждое ребро либо насыщается, либо не насыщается. Но так как 
        ребро насыщается не больше 1 раза, а если ребро не насытилось, то дефицит потока
        был компенсирован и вершина сменилась, то при каждом проталкивании будет сделано не больше 
        $|V|$ ненасыщающих минипроталкиваний, и $|E|$ насыщающих на всей итерации поиска блокирующего 
        потока.
     \end{enumerate}
     Итоговая асимптотика $O(|V| * (|V| + |E| + |V| * (|V| + |V|) + |E| = O(|V| * (|V|^2 + |E|)) = O(|V|^3)$
\end{document}
