\documentclass{article}
\usepackage[a4paper, margin=2cm]{geometry}
\usepackage[utf8]{inputenc}
\usepackage[russian]{babel}
\usepackage{mathtools}
\usepackage{amsmath}
\usepackage{amsfonts}
\usepackage{tikz}
\usepackage{amsthm}

\usetikzlibrary{automata,positioning}

\newcommand{\range}[2]{
    \{#1, \ldots, #2 \}
}

\title{Домашнее задание 8}
\author{Макаров Михаил}
\date{}

\newtheorem{prop}{Утверждение}

\begin{document}
	\maketitle
    \section*{1} 
    Так как $e^x$ монотонно возрастает на $\mathbb{R}$, то для любого $q \in \mathbb{Q}$ существует не больше
    одного корня уравнения $e^x = q$, следовательно множество $A$ не более чем счётно, следовательно оно измеримо 
    по Лебегу и его мера равна $0$.
    \section*{2}
    Если $A \in [0, 1], X \in [0, 1]$, то $ A \Delta X = ([0, 1] - A) \Delta ([0, 1] - X)$. \\
    Зафиксируем $\epsilon > 0$, 
    тогда $\exists A: \mu^*(A \Delta X) < \epsilon \leftrightarrow \exists A: \mu^*(([0, 1] - A) \Delta ([0, 1] - X)) < \epsilon$, следовательно $X$ измеримо по Лебегу если и только если $[0, 1] - X$ измеримо по Лебегу.
    \section*{3}
    Пусть $E = \bigcup_{i = 1}^{\infty} E_i$, $F = \bigcup_{i = 1}^{\infty} F_i$. 
\end{document}
