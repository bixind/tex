\documentclass{article}
\usepackage[a4paper, margin=2cm]{geometry}
\usepackage[utf8]{inputenc}
\usepackage[russian]{babel}
\usepackage{mathtools}
\usepackage{amsmath}
\usepackage{tikz}
\usepackage{amsthm}

\usetikzlibrary{automata,positioning}

\newcommand{\range}[2]{
    \{#1, \ldots, #2 \}
}

\author{Макаров Михаил}
\date{}

\newtheorem{prop}{Утверждение}

\begin{document}
    Назовём пару индексов $l$, $r$ верхней полусосиской, если существует такая позиция $u, (l \leq u \leq r)$, 
    что подмассив $a$ с $l$ по $u$ не убывает, подмассив $a$ с $u$ по $r$ не возрастает.
    Заметим, что если $l, r$ --- верхняя полусосика, то $l, i, (l \leq i \leq r)$ --- тоже верхняя
    полусосика.

    Определим $c[l] := max\{m| \text{$l, (l + m - 1)$ --- верхняя полусосика} \}$. 
    Заметим, что если $a[l] > a[l + 1]$, то любая 
    полусосиска с правым концом $l$ будет невозрастающим подмассивом $a$. Найдём для каждой позиции $h[i]$ ---
    максимальная длина убывающего подмассива с началом в $i$, например с помощью обратной динамики по массиву.
    Тогда если $c[i] = m$, то $\exists i: h[i] = k$ и $a[l], \ldots, a[l + m - k]$ - неубывающий подмассив.
    Таким образом значения $c$ можно найти аналогично поиску $h$, 
    учитывая, что длина неубывающего подмассива может быть равна 1.
    
    Аналогично можно определить нижнюю полусосику и $d[l] := max\{m| \text{$l, (l + m - 1)$ --- нижняя полусосика} \}$.
    Теперь длина сосики нетрудно ищется как $max\{d[i] + c[i]| 1 \leq i \leq n \}$.
\end{document}
